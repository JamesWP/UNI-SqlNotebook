\documentclass[a4paper, 11pt]{article}
\usepackage{comment} % enables the use of multi-line comments (\ifx \fi)
\usepackage{fullpage} % changes the margin
\usepackage{csquotes} % for quotations
\usepackage{hyperref} % for hyperlinks

\begin{document}

\noindent\large\textbf{3rd Year Project Report} \hfill \normalsize Project Supervisor  \\
\textbf{James Peach}  \hfill
\textbf{Dr. Alvaro A A Fernandes}\hfill

\section*{Abstract}

\section*{Acknowledgements}

\section*{Table of contents}

\section{Introduction and Motivation}

There are lots of ways to write programs. And while working in the real world
you get to do things in lots of ways. Some of the ways work well, and others
leave things to be desired. In this section I will outline how I came  to pick a
project where I am tasked to creating a new way to manage Sql database projects.

Over the last few summers I had the opportunity to work on various applications
for a small business. The applications were all based on a Sql database as the
store for all the logic and data. Over my time spent developing and maintaining
the applications, I found increasingly that I was having to rediscover parts of
the applications when I was forced to switch between different tasks. This
causes more mistakes to be made by the programmer. I needed to know how
component $a$ was connected to component $b$, what the field was in table $x$
that stored the name of the employee's grandfather. I wanted a way that would
allow me to program the database and document it at the same time or at least
have the correct information at my fingertips.

Over the next few months working I was looking at different tools and languages
and stumbled across other students using $IPython Notebook$. For those who don't
know what the $IPython Notebook$ is, by the project description:
\begin{displayquote}
The IPython Notebook is an interactive computational environment, in which
 you can combine code execution, rich text, mathematics, plots and rich media
 \cite{IPython Notebook}
\end{displayquote}
The IPython Notebook is web based, and accessed via a browser and allows the
code to live interleaved between the text and can provide a nice way to store
extra information about the code and descriptions of the different parts of the
database.

When I originally saw the $IPython Notebook$ I thought I was just going to
implement a similar version for Sql, however I soon found myself extending this
initial idea with thoughts to a more rounded project. I wanted to have a
complete, working, and most importantly helpful system for managing all the
knowledge required to develop day to day in a Sql database project.

I started with identifying the type of person or user of such a system and what
their tasks might be. I constructed a few ``Day in the life of..." scenarios
would each identify a person trying to accomplish a task and how they go about
it. The difficulty in creating these is in being too specific and not correctly
capturing the tasks would lead to a solution that would only be useful in
some tasks.

\begin{itemize}
  \item{Development}
  \item{bugs}
  \item{reports}
  \item{performance}
  \item{analysis}
\end{itemize}

\section{Background}

There will be a detailed explanation of the tools available in the market
and what each individual tool brings to the area.

\begin{itemize}
  \item{Command line interfaces}
  \item{Sql-Server Management studio}
  \item{PhpMyadmin}
\end{itemize}

\section{Design}

I will then give a high level explanation of the project and its goals with an
idea on how my project might solve them.

\begin{itemize}
  \item{to manage documents and code}
  \item{to encourage up to date documentation of the system}
  \item{to help the developers understand what the code does not explain}
  \item{to make simple tasks easy, to put emphasis on the harder tasks}
\end{itemize}

I have made various choices during the development of the project i will explain
what each of the main choices were and what my thoughts on them are. I
specifically want to be able to reason about the choice of medium (the web) and
weigh up the pros and cons

\begin{itemize}
  \item{medium}
  \item{development style}
  \item{use of metaphor}
  \item{usefullness of features}
  \item{technology}
\end{itemize}

Here i will have a section with the final list of formal requirements for the
project

\begin{itemize}
  \item{}
\end{itemize}

\section{Implementation}

In this section i will give an overview of how i intended to go about
implementing the system (non technical).

\begin{itemize}
  \item{Idea reasearch}
  \item{identify requirements}
  \item{notebook metaphor: inc. links, index search etc..}
  \item{technology research and trial}
  \item{testing strategy}
\end{itemize}

In this section there will be a paragraph on the challenges with adopting a new
technology (react.js) and how this affected the project development.

\begin{itemize}
  \item{speed of coding}
  \item{seperation of concerns}
  \item{how react.js might affect the testing of the project}
  \item{build process}
\end{itemize}

\section{Testing}

In this section i will begin by explaining some testing methodology and the
challenges with testing a GUI based project.

The project itself is an exploration into a new interface for dealing with
database systems. I.e. more of an exploration into user experience

\begin{itemize}
  \item{unit testing of modules}
  \item{automated browser testing}
\end{itemize}

\section{Conclusions}

Talk about the idea of the project and how it has developed.
Evolution into an information database.

Things complete and things not yet complete.

Self Evaluation of the finished project with regards to the requirements set out

Things learnt from taking on the project

Future improvements

\begin{itemize}
  \item{security}
  \item{extend to multiple users}
  \item{host storage online}
\end{itemize}

\begin{thebibliography}{9}

  \bibitem{IPython Notebook} IPython development team \emph{The IPython Notebook -
  IPython}. Available from
  (\href{http://ipython.org/notebook.html}{http://ipython.org/notebook.html})
  accessed 27th December 2015

\ifx
  \bibitem{Robotics} Fred G. Martin \emph{Robotics Explorations: A Hands-On Introduction to Engineering}. New Jersey: Prentice Hall.
  \bibitem{Flueck}  Flueck, Alexander J. 2005. \emph{ECE 100}[online]. Chicago: Illinois Institute of Technology, Electrical and Computer Engineering Department, 2005 [cited 30
  August 2005]. Available from World Wide Web: (http://www.ece.iit.edu/~flueck/ece100).
\fi

\end{thebibliography}


\end{document}
