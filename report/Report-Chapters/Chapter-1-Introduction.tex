\chapter{Introduction}\label{introduction}

Developers use documentation to aid their understanding of code, however
the documentation is often constructed and viewed on paper or in a web
browser
\href{http://www.oracle.com/technetwork/articles/java/index-jsp-135444.html}{for
example, (cite javadoc here)} While the use of documentation tools
provides a good overview of the code, it is not the be all and end all
of the notes surrounding the code. Developers can't rely on the
documentation alone, they must also have an understanding of the other
components of the system and their interaction.

This is also true of database applications. The database applications
are often the store for the information in the system and provide a
center for the business logic for the application. They often are at the
center of large applications and as a result they can be complicated.

It is therefore important that the developers that develop and manage
these applications know exactly the effects of the code and what there
changes mean in terms of the system as a whole.

\section{Project goal}\label{project-goal}

This project's goal is to create an application that will help database
developers to document and code in a more consistent and interactive
manner.

The project will cover methods used for identifying problems with
existing applications and workflows and then formalize these problems
and aim to provide an evaluation of the different approaches to solving
them along with reasons for selecting the final approach.

\section{The users}\label{the-users}

In order to understand the application we first need to understand the
users of the application. These users are developers. The developer's
days could consist of many different types of tasks, from developing new
applications, new features, and maintaining old applications and
services. These are just a few of the most common tasks required.

This poses problems when creating software for these developers to use;
while they require places to document code and other items, they don't
do so in the same way every time or indeed do so with a regimented
structure. The tools they use must support their individual ways of
working. However they do still need ways of helping them to manage the
information.

Creating an application for this purpose will require considering the
many problems that existing applications in the market-place have
attempted to tackle and how successful their attempt was.

One example of the type of problems an organization application would
look to solve is this: A developer might be working on more than one
application and also be responsible for supporting older applications.
When the developer receives notice that there is a problem with another
project the current work needs to be parked in a suitable place and the
developer goes in search of details about the system newly in question.
Making this process of switching projects and finding the required
information quickly is one that this project attempts to tackle.

The problem above that has been identified is just one problem and there
are many more that would need to be identified. This report details how
these have been found and their solutions designed and built.

\section{Overview of the report}\label{overview-of-the-report}

This report explains how the final solution was developed from an idea
into the final product. It does so with some guidance from standard
software development techniques. First some problems are identified then
a formal set of requirements is derived. These requirements guide the
design of the application and after the application is built the
requirements are again used to verify that the end product does indeed
solve the original problems identified.

The problems that the application hopes to address will be a collection
of things that occur in every day life as a developer and other problems
identified with existing tools available. This is done so that the
developers can see the main problems with the existing tools can indeed
be fixed and alternatives can be used.
