\chapter{Introduction}\label{introduction}

Developers use documentation to aid their understanding of code, however the
documentation is often constructed and viewed on paper or in a web browser. An
example of this for Java applications is \textit{Java-doc}\cite{javadoc}. While the use
of a documentation tool provides a good overview of the code, it shouldn't
exist in isolation.  Developers can not rely on the documentation alone, they must
also have an understanding of the other components of the system and their
interaction.

This is also true for database applications. Database applications are often the
backing store for the data of the system and also provide a centre for the
business logic for the application. They are often at the centre of large
applications and as a result they can be complicated. This is because databases
are designed to store and validate large amounts of data. They also provide many
useful mechanisms for integrating with a large number of other technologies.

It is therefore important that the developers that develop and manage
these applications know exactly the effects of the code and what their
changes mean in terms of the system as a whole.

\section{Goal}\label{project-goal}

The project's goals are to create an application that will help database
developers document and code in a more consistent and interactive
manner. The result should try to accomplish this within one tool, both for coding, and for documenting.

The project will cover methods for identifying the problems with existing
applications, and formalising them. The project then aims to provide an
evaluation of the different approaches to solving the problems along with
reasons for any choices made.

\section{Users}\label{users}

In order to understand the application we first need to understand the
users of the application. These users are developers. The developer's
days could consist of many different types of tasks, including developing new
applications, new features, and maintaining old applications and
services. These are just a few of the most common tasks required.

This poses problems when creating software for these developers to use; while
they require places to document code and other items, they do not do so with a
common structure. The tools they use must support their individual ways of
working. However they still need ways of helping them to manage the
information.

Creating an application for this purpose will require the consideration of many
problems that existing applications have attempted to tackle and their success.

One example of the type of problems an organisation application would look to
solve is this: A developer is working on more than one application and is
responsible for supporting older applications. The developer receives notice
that a problem exists with a project he worked on previously. His current
progress needs to be noted so he can continue at a later date. The developer
then must find details about the problem. The process of switching projects and
finding the required information is a problem this project aims to solve.

The problem above is just one problem and there are many more that would need to
be identified. This report details how they were found and their solutions
designed and built.

\section{Overview of the report}\label{overview-of-the-report}

This report explains how the final solution was developed from an idea
into the final product. It does so with some guidance from standard
software development techniques. First, some problems are identified, then
a formal set of requirements is derived. These requirements guide the
design of the application and after the application is built the
requirements are again used to verify that the end product does indeed
solve the original problems identified.

The problems that the application hopes to address will be a collection
of things that occur in every day life as a developer and other problems
identified with existing tools available.
