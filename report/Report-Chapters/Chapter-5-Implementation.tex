\chapter{Implementation}\label{implementation}

This chapter in this report will detail how the application was
implemented given the choices made in the Design chapter and the
requirements. The application was developed in the agile methodologies.
Quick and small iterations where a small part of the application will be
planned, implemented and tested. By ensuring the parts are kept to a
consistent small size, the development should be kept going at a
methodical pace.

As always with development projects involving code version control is an
important part of the development process. For this project git has been
used to keep track of the changes to the code. This provides many
benefits as can be seen in
\href{https://books.google.co.uk/books?hl=en\&lr=\&id=aM7-Oxo3qdQC\&oi=fnd\&pg=PR3\&dq=version+control+software\&ots=38zcOCRfo9\&sig=RrjwXCtjK_7I4AtQZiOWj5wSo5Q\#v=onepage\&q\&f=false}{this
book} published on git.

Keeping the iterations small and using technologies like git will enable
me to be agile when developing. Being agile means that I can react to
complications with development and delays that occur due to new
languages and technologies.

With this in mind the application was broken down into smaller parts
that fit together to form the application. The front end of the system
is the part the users will directly be interacting with. This would form
the bulk of the development however, the backend of the application is
the middleware component that will enable the javascript front end to
talk to the database.

\section{Middleware}\label{middleware}

The middleware was planned to be created first. It was a simple task and
was well defined. The requirements for this part of the application were
as follows:

\begin{itemize}
\tightlist
\item
  Allow the caller to create a connection to a database.

  \begin{itemize}
  \tightlist
  \item
    This will require the caller to pass connection parameters and in
    return they will receive some connection token to represent the
    connection.
  \end{itemize}
\item
  Allow the caller to execute queries against a connected db.

  \begin{itemize}
  \tightlist
  \item
    Given the connection token received in the above step and a query,
    the results of the query need to be sent back to the caller.
  \end{itemize}
\item
  Allow the caller to receive context data regarding a query.

  \begin{itemize}
  \tightlist
  \item
    Similar to executing a query, the application might need to receive
    details about the objects mentioned in the query. Each object
    mentioned in the query, table, view, function etc\ldots{} would be
    explained in the response.
  \end{itemize}
\end{itemize}

\subsection{Design the API}\label{design-the-api}

The communication to the database is provided by use of the
SqlConnection and SqlCommand classes provided by C\#. To execute a query
you must first create a SqlConnection class with the connection string.
A connection string is a textual detail of the parameters required to
connect to the database. This can contain many different types of
connection however the more usual address and username and password will
be used to connect to the database in the api.

\begin{verbatim}
    public class ContextRequest
    {
        public string Database { get; set; }
        public string Server   { get; set; }
        public string Username { get; set; }
        public string Password { get; set; }
    }
\end{verbatim}

The ContextRequest class provides the definition for the parameters for
the CreateExecutionContext when the /api/CreateExecutionContext method
received a http post the parameters are extracted from the body of the
post and a sql connection is created, opened and a token is returned to
the user.

From javascript there are many different ways in which you can do a HTTP
post. The library that is used in the application is called superagent
\emph{cite superagent} this provides a set of functions that manage HTTP
posts. Here is an example of how a HTTP post can be made to \emph{url}
with parameters \emph{params}. When the post has completed the callback
function is called with the response.

\begin{verbatim}
  request
    .post(url)
    .send(params)
    .type('application/json')
    .accept('application/json')
    .end(newCallback(callback));
\end{verbatim}

Here the response is the connection token. This token will be needed to
execute all further requests to this connection.

\section{Javascript application}\label{javascript-application}

The javascript application provides the user interface for the
application. It is written in javascript and runs in the browser in a
single web page. Requests to the SQL middleware API are made over HTTP
with the superagent package.

Applications built for browsers can have very large codebases, There are
many different problems that arrive from having a large code base and
there are tools available to help developers manage the applications.
With standard javascript there is currently no module system (although
this may change soon \emph{cite ES6 modules}) and so if we want to split
our application into multiple smaller components we must ensure that
they are all correctly loaded in the browser before being required. This
can become difficult with many interdependent modules.

There are ways to automate this process and automatically find all
dependancies and bundle them all into one file that the browser can
load. \href{http://browserify.org/}{Browserify} is one such program.

\begin{verbatim}
Browserify lets you require('modules') in the browser by bundling up all of your dependencies.
\end{verbatim}

This enables developers to think in smaller files and have the
dependancies and load order be solved by a tool.

The Browserify process can also be extended to optimize our javascript
code and also provide us with analysis to point out potential bugs and
also best practices. These tools help to turn the javascript development
process into something similar to ahead of time compiled code like C\#
and Java and C++. This is needed for larger applications to be built in
the browser.

\subsection{React.js development}\label{react.js-development}

With React.js the UI is defined by React components. Each component can
be thought of as a new html element. They can have attributes defined:

\begin{verbatim}
  <NewElement attributeNo1="Ted" attributeNo2="123.4"/>
\end{verbatim}

And each component has a render function, this render function takes
each components internal state and produces more html for the page.

\begin{verbatim}
  var HelloMessage = React.createClass({
  render: function() {
    return <div>Hello {this.props.name}</div>;
  }
  });

  ReactDOM.render(<HelloMessage name="John" />, mountNode);
\end{verbatim}

In the above example the HelloMessage component is created. In this
simple component when it is used it outputs just a simple html Div
element.

The use of React.Js in the project forces structure in the use of
components. The layout of the application is as follows: - controllers/
- helpers/ - stores/ - ui/ - app.js

The application is created in the app.js and this then uses the
components in the UI folder to create the interface. The app.js file
completely separates the UI into the binder (the left hand side of the
interface) and the workspace (where pages are opened).

\subsection{Storage of data}\label{storage-of-data}

The application will store the data for the open notebook in the browser
until it is closed and saved to disk. The data in the browser will form
the state of the react components and any changes to this state will
cause React.js to re-draw the ui.

The pages are stored in the PageStore module. The tabs are stored in the
TabStore module and the workspace in the WorkspaceStore module. Each of
the stores contains the current state of the objects and components that
need to access data from them add listeners when created and receive
events when things change. This way we can separate the code that stores
the data and modifies it and the code that creates the ui. This
separation of concerns simplifies the details the developers needs to be
aware of when writing code.

\subsection{Data store interface}\label{data-store-interface}

The interface that the application uses to change the data in the store.
It is important that this interface is as simple to use as possible in
order to ensure that all the complicated logic to change the data is
completely within the store.

First the actions were planned out that the user might trigger. - Create
new page - Save page - Delete page - Create new tab - Change search text
- Connect / Execute query

These actions form events on the data stores and update the internal
state. The changes in the state then cascade down to the components that
depend on them and update the UI.

\subsection{Components}\label{components}

The application was split into small functional parts and a high level
design for the application was drawn out.

\emph{diagram}

The left side of the application was dedicated to the navigation
interface with tabs and pages displayed with buttons to create new and
remove old.

The workspace is a container for the open pages in the application each
page is a different component depending on what type the page is. For
example a search page renders a Search component and an index page
renders an Index component. This allows for easy addition of new window
types to the system as each is implemented in a separate file.

The initial design was implemented with the standard html entities li
for list and button for actions. Later on in the design process these
were replaced with components form a third party library of elements
that provided sections of
\href{https://design.google.com/spec/}{material design} this proved the
use of Browserify and React.js because the third party library
\href{http://www.material-ui.com/\#/}{material ui} could be just
referenced and the components used to replace the default html list
elements and buttons etc and the application ui was improved.
