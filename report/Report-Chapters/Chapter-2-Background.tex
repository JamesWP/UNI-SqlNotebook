\chapter{Background}\label{background}

What is the need for documentation and notes?

There are many different ways to document code they range from
completely offline to completely digital and many different methods for
integrating code. The most simple offline approaches consist of simple
notepad pages with notes and diagrams. This offers a totally freeform
approach to note taking as with pen and paper there is no limit to what
you can make notes about. Tables diagrams and hand written pseudo code
can be good for some circumstances but for others we have digital
documents that need to form part of the notes along with written
comments, This leads to problems of printing and loosing loose pages.

There are also many solutions for electronic or digital notes, those
taken on a computer. \emph{cite Evernote}
\href{https://evernote.com}{Evernote} is a popular choice for notes and
offers a wide range of platforms for note creation however it is not
designed with code documentation in mind specifically. While it is a
very good solution for taking notes as is evident in the number of users
\emph{cite Evernote success}. the lack of specific features aimed at
developers eg version control etc makes it difficult to use in this way.

\begin{quote}
JP how to cite this?
\end{quote}

The above solutions are not tailored to suit the specific task of making
notes for sql developers. The existing tools used by developers can be
used in some capacity to both develop new code and help document it for
future readers. Commenting code is often the closest notes get to the
code they are detailing. Well commented code can help developers get to
know and remember the code they were writing in the past or to pass on
details from programmer to programmer. Comments on code can often not
last the test of time \emph{cite example of out of date comments} as
they don't force you to update your note when the code they refer to
changes. For example over time the commented function might offer more
options and change from its original purpose. This is the reason
comments are not always taken to be the truth and can lead to confusion
and bugs in the worst case. \emph{find example of out of date comment}

\section{Analysis of existing
solutions}\label{analysis-of-existing-solutions}

The different applications on offer vary from exact programming language
and operating system. They have useful features and also each have there
own downsides.

Analyst different solutions for SQL IDE's and also note taking
applications. drawing positives and negatives from each.

\subsection{Microsoft SQL Server Management
Studio}\label{microsoft-sql-server-management-studio}

\emph{citation needed}

\emph{screenshots needed}

An official offering for the Microsoft SQL Server Database server. Has
the ability to have multiple connections to servers and many open files
in an IDE powered with the Microsoft Visual Studio Shell. Offers little
to no management of files other than the ability to open files from the
filesystem. As with many database servers the functions and procedures
are stored within the databases and so any files on the local system
might not be in sync with the ones on the server. The application is a
native windows application and offers no alternatives for other
operating systems. The connections offered are the same as for the
database itself via TCP, Shared memory, named pipes etc. This poses many
problems for working when not on the same network and forces the users
to be using Microsofts own operating system.

\subsection{MySQL Workbench}\label{mysql-workbench}

\emph{citation needed}

\emph{screenshots needed}

The MySQL Workbench offers a multi-platform solution for accessing a
MySql database. Also on offer is a model management tool for mapping out
a logical model of the database and relations this provides a way to
easily view a high level overview and make useful annotations that go
above and beyond simple comments written separately.

The Workbench also is the first example of a tool to offer
multi-platform capability. Providing more ways for the users of an
application helps integrate a tool into the workflow of the developer,
having to switch between operating systems because your database
management software is not offered on the same work machine as your
front end development environment

\subsection{phpMyAdmin}\label{phpmyadmin}

\emph{citation needed}

\emph{screenshots needed}

The alternative to the official MySqlWorkbench is an open source web
based solution for easy access to the database from anywhere you have an
internet connection. The tool is a PHP application and such needs a
hosting environment in order to be used. The application, once running
has a secure login system with the ability to view and edit data in the
system.

However the tool offers little in the way of editing for scripts and
database code. Bulk manipulation of data is not offered in the
application but instead through the export of multiple different file
types.

Other unique features are automatic linking of data through the
interpretation of foreign keys within the existing database.

Unlike the MySql Workbench there is no designer for models and all
changes are done to the database ``live'' there is no design and publish
as offered in other solutions.

\subsection{IPython / Jupyter}\label{ipython-jupyter}

\emph{citation needed}

\emph{screenshots needed}

This application differs from the others above as it is not a SQL IDE
and such it doesn't show any specific features for SQL code in
particular however the integration between the code and the text in the
editor is done in a very easy to use way.

The IPython notebook is split into sections or cells. each cell has a
type, either a code cell or title or markdown cell. Markdown is a simple
written markup language for textual documents. Markdown provides simple
ways to create headings and lists paragraphs while balancing this with
simple easy to read minimal formatting.

\begin{verbatim}
# Heading level 1
Paragraph content here

### Heading level 3

Below is a list
* lists are created
* like this
* or numerically

### new list
1 like
2 this
\end{verbatim}

The mix of executable code within the text notes provides a unique
experience of the ability to see and tweak code and have a well
formatted explanation of the code.

Interactive code and results as done in an IPython notebook provide a
quick way to see the results of the code. This allows and promotes the
testing and exploration of the code and can help the developer to
understand exactly what effects each part of the code has on the end
result.

The notebooks reside on files on the local disk of the system, much like
any other file. As such the organization of the files is left up to you
the creator to file them in folders etc.

The flexibility comes at a cost however, because while you can file the
notebooks anywhere you want, where to file them is done by you in your
file manager.

\section{Problems}\label{problems}

\subsection{Problem 1 Keeping notes of important
things}\label{problem-1-keeping-notes-of-important-things}

As developers work on their projects and as these projects evolve there
become an increasing number of things that need to be understood in
order to develop / maintain the project. The problem is to provide an
easy flexible way to make notes on different topics and allow access to
the notes in an organized way.

\subsection{Problem 2 Notes are only useful when they are
read}\label{problem-2-notes-are-only-useful-when-they-are-read}

The notes / code stored need to be accessible for developers or this
defeats the purpose of storing the information in the first place.

The notes as discussed don't have an implicit structure and so a helpful
search is difficult to create. Most solutions just offer full text
search of the data.

The notes might have had several revisions over time and might need the
information accessible from the past.

\subsection{Problem 3 Notes that include code are often not
updated}\label{problem-3-notes-that-include-code-are-often-not-updated}

The notes that store code are often not in an executable format. so when
the dependancies of the stored code are changed the code becomes
outdated and of limited use to future developers.

This is a problem because the code is only useful in a note when it is
up to date and bug free, often at the time of writing it is correct but
as the system changes the code can become out of date.

The problem is how to ensure or help the user to keep this code up to
date within the note.

\subsection{Problem 4 Database systems: data always
changing}\label{problem-4-database-systems-data-always-changing}

The data in database systems is always changing, when the system was
written the data might have looked significantly different to what it is
now. This makes things hard to compare as time goes on.

The problem is how to ensure that even if the underlying data is
changed, the note and code within is still useful for other developers
in the future.

\subsection{Problem 5 System access}\label{problem-5-system-access}

The developers need easy access to the information and the database from
anywhere. Current solutions for IDE's are often system specific and
don't offer good cross platform compatibility.

The problem is providing access to the data and system from whatever
system, and wherever the developer is in a secure way.

\subsection{Problem 6 Cross
referencing}\label{problem-6-cross-referencing}

The document writers are writing multiple notes for each part of the
project and as such there will be links within the document by way of
referring to something written elsewhere. The ability to refer to other
sections in other documents is key in order to provide the reader all
the information in order to understand the page.

\section{Summary of identified
problems}\label{summary-of-identified-problems}

The problems identified above fall into different categories:

\begin{itemize}
\tightlist
\item
  organization of notes
\item
  sql specific problems
\item
  finding relevant notes
\end{itemize}

Each of the above categories of problems could be tackled in many
different ways. The next stage in the development of a complete solution
is to identify some concrete requirements and use cases for the
application and discuss the different approaches available to us in
order to solve them. Each different approach will have its own downsides
and these need to be properly analyses and evaluated if a useful
solution is to be created.
