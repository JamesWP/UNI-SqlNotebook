\chapter{Design}\label{design}

The design of the application will draw from the above requirement and
the associated use cases to form a final application. There are many
decisions that impact the delivery of the final project they are details
in this chapter and the respective different options are detailed and
evaluated.

\section{Structure of the information in the
solution}\label{structure-of-the-information-in-the-solution}

The application will store the notes and code of the developers, however
there are many different ways to store these files. Here is a few of the
different ways to store this information with pros and cons for each:

\subsection{List of files}\label{list-of-files}

The application could store the notes as simple files in the system in a
list. Similar to a flat single folder structure. Each file would have a
name and other attributes like last edited. This method is the most
simple of the below however it requires the user to name the files with
discipline if they want it to remain organized. For example the
following is an idea of how this may look.

\begin{verbatim}
    - Project A - Overview - Project summary
    - Project A - Overview - Change log
    - Project A - Users - User login module
    - Project A - Users - User login module version 2
    - Project B - Overview - Project summary
    - ...
\end{verbatim}

This is not an ideal way to view files for a single section of a system.
Typically you are working with a single project at once and within that
usually not all parts but a limited subset. This problem could also be
viewed as a positive because in the times that you do need to access
files across multiple projects at once, i.e.~in the use case when the
developer needs to switch to another project quickly, the developer can
already see the entire list of files.

\subsection{Folder structure, i.e.~Explorer /
Finder}\label{folder-structure-i.e.explorer-finder}

There are already existing methods for storing files into folders or
directories. Most mainstream operating systems use a hearty of folders
to contain files and more folders. This acts like a tree, each leaf is a
file and each subtree is a folder.

This approach still offers the users free reign over the organization
and naming of each file in the system however it also provides the
ability to focus on single parts of the tree at once.

\begin{verbatim}
- Project A/
  - Overview/
    - Project summary
    - Change log
  - Users/
    - User login module
    - User login module version 2
- Project B/
  - Overview/
    - Project summary
\end{verbatim}

This organization into folders can take time to do and can be done in
many ways. Each project might have its own collection of top level
folders and in each project (unless managed correctly) there could exist
a different way of creating / naming the folders. This problem of
uniformity crops up in many places within organization and in general
the more uniform the structure the less the user has to remember for
each case and the less brain power needs to be spent understanding it.

There is another problem with a simple tree folder structure, each file
in the folder tree can only be in one place at once. Some files could be
ambiguously belong in multiple places at once. This can make finding
these files take twice as much time. Each ambiguously located file that
needs to be found needs to be searched for in (worst case) all the
locations it could be in.

Both the two above problems are lessened by putting sensible limits on
the depth of the folder structure. The reason for this is discussed
below.

\subsection{Simplified folder
structure}\label{simplified-folder-structure}

In a simplified folder structure there is a limited number of levels
that can exist. If viewed as a tree it is a tree that contains a limited
depth and contains leaf nodes (here notes) at the bottom level of the
tree.

This simplified structure can help make the choice of folders at each
level more apparent for the user. They have a more limited choice of
where to place folder and files and as such the impact problems of a
full folder structure as discussed above are lessened.

This technique also implicitly limits the user to a set number of
levels. This (although simplifying) imposes that some files that might
have been better suited to live together inside a folder have to be
placed within the parent folder.

\subsection{Tagged files}\label{tagged-files}

In a tagged file organizational system, files are not placed in folders
but instead have each an associated set of ``tags'' each of these tags
has a name and can be applied to many files.

For an example you might have the following set of files and their tags:

\begin{verbatim}
- Project summary {Project A, overview}
- User login module {Project A, Users}
- Overview {Project A, overview}
- Project summary {Project B, overview}
\end{verbatim}

The user can then find the subset of files they are interested in by
filtering for the items that have all the tags they specify:

\begin{verbatim}
Filter: file has all {Project A, overview}
Result:
- Project summary {Project A, overview}
- Overview {Project A, overview}

Filter: file has overview
Result:
- Project summary {Project A, overview}
- Overview {Project A, overview}
- Project summary {Project B, overview}
\end{verbatim}

Note that this can be viewed as a more general version of the folder
structure discussed in section \emph{insert folder structure section ref
here} by adding a single tag to each file with the path of the folder
that contains it. However this is not an ideal tagging for files

Citation:

\begin{verbatim}
@article {ASI:ASI22906,
  author = {Bergman, Ofer and Gradovitch, Noa and Bar-Ilan, Judit and Beyth-Marom, Ruth},
  title = {Folder versus tag preference in personal information management},
  journal = {Journal of the American Society for Information Science and Technology},
  volume = {64},
  number = {10},
  issn = {1532-2890},
  url = {http://dx.doi.org/10.1002/asi.22906},
  doi = {10.1002/asi.22906},
  pages = {1995--2012},
  keywords = {personal information systems},
  year = {2013},
}
\end{verbatim}

\subsection{Choice of file
organization}\label{choice-of-file-organization}

The simplified folder structure was picked because of the balance
between the flexible organization of the tagging system and the
simplistic file list. The simplified folder structure proved to be
simple to implement and use within the system

\emph{images of the final version}

\section{Search capability in the
system}\label{search-capability-in-the-system}

The requirements gathering process clearly identified in multiple places
the need for the system to provide a quick and simple search function in
order for developers to find the files that they need.

There are multiple ways to provide search functionality within a
document management system some of which are suitable for the
application and some that are not. The below is a summary of a few of
the options and an evaluation of each.

Ideally the functionality required is that a user can submit a query for
some content, the files in the system is searched for the specified
items and the results are shown. This part of the searching experience
will be the same for all the discussed methods. However the parts that
differ in the compared methods are: * the format of the users query
(plain text or other). * the searching procedures * the performance *
the presentation of results

\subsection{Simple text search / serial
scanning}\label{simple-text-search-serial-scanning}

The simplest method for searching is taking a simple section of text
from the user, the search term, and then searching for the occurrences
of the term in each file and for each result displaying the name of the
file, and the position within the document.

Example:

\begin{verbatim}
  search: "test"
  results:
  - File A, line 13, chars 34-38
    - ... the users will then report on *test* ing ...
  - File A, line 13, chars 10-13
    - Chapter 2: *Test* ing ...
\end{verbatim}

This technique is simple and for small amounts of files and content is
quick to compute. However due to the fact that the algorithm is loosely
linear in the number of files to search through, this would be
impractical for large amounts of files.

\emph{cite wikipedia:
https://en.wikipedia.org/wiki/Full\_text\_search\#Indexing} When dealing
with a small number of documents, it is possible for the
full-text-search engine to directly scan the contents of the documents
with each query, a strategy called ``serial scanning.''

\subsection{Full text index}\label{full-text-index}

When a simple method for searching through all the files becomes too
slow, an alternative is to create a concordance \emph{cite
https://en.wikipedia.org/wiki/Concordance\_(publishing)} this is a table
of words in a publication (here note) and where the words appear within
the document. by using this concordance instead of searching the text
directly, the application is able to search a smaller amount of data and
hence speed up the search process.

This requires that there is a pre-processing step before a search can be
completed. This needs to be recalculated for each time a user changes a
document so that all subsequent queries can find the new content.

This is the tradeoff with the more complex methods for searching.

\subsection{Extraction of keywords}\label{extraction-of-keywords}

Another method of searching is by analysis of keywords only. Indexing a
select few words per note means smaller indexes and faster searches.
Either a list of keywords is created and left static or machine learning
algorithms learn the important words within the page and only those
words are indexed.

This is however only applicable when the user's search terms can be
limited as such. For example if the user searches for a table name that
is contained within a file in the application, it will only be found if
the table name is in the keywords list. In the case of a note taking
application there is limited use of a search process that only finds
some words of the documents stored. However such indexes can help to
find preliminary results for a query, or even to provide caches of
common queries.

\subsection{Edit distance}\label{edit-distance}

\begin{quote}
JP maybe discuss fuzzy search
\end{quote}

\subsection{Reasons for selecting simple text
search}\label{reasons-for-selecting-simple-text-search}

The numerous ways to search the text within a collection of documents
each provide there own ways to provide the results. While many of the
different search methods are concerned with the performance of a search
and their accuracy in terms of the results, the most basic search
discussed the Simple text search \emph{link above section} as discussed
can offer the features needed by most users of the application.

There are many other features that can be added to the functionality
provided by the Simple text search \emph{link} however these can be
added as an extension to the program after the initial release.

\section{Type of user interface}\label{type-of-user-interface}

The different types of user interface that designers can choose from
when designing applications can be categorized into a few main
categories:

\begin{verbatim}
- Command line
- GUI / WIMPS
\end{verbatim}

For visualizing text, popular command line applications for editing text
like Vim or Emacs do exist however, the richer environment of a GUI
application is predominately preferred with its ability to customize
fonts and other such attributes of the layout and style. They also
provide a better experience for users with the use of menus etc. These
can provide a more simple to use application than an alternate command
line program with many shortcuts to memorize.

There are many different types of GUI application that can be built and
an important decision that needs to be made before selecting the layout
of the application is what platform to build the application on.

\subsection{Native solution for operating
system}\label{native-solution-for-operating-system}

The first platform option is developing a desktop first application
native in a selected operating system. This would require selecting one
of the native windowing API's for the operating system. Some of these
API's are cross platform enabled for example see
\href{http://www.qt.io/}{qt} for a popular choice. Some more polished
and more integrated with the selected operating system for example
Apple's own windowing API is
\href{https://en.wikipedia.org/wiki/Cocoa_(API)}{Cocoa}

The advantages of using the operating system's manufacturer built API is
one of support from the company and choice of open source vs
proprietary. As one of the requirements is that the system should be
accessible the proprietary method is not an option for a time
constrained development like this. In order to complete a cross platform
version using operating system specific (non cross platform) API's would
mean a rewrite of the GUI code for each operating system that was to be
supported. The advantages of using the operating system's manufacturer
built API is one of support from the company and choice of open source
vs proprietary. As one of the requirements is that the system should be
accessible the proprietary method is not an option for a time
constrained development like this. In order to complete a cross platform
version using operating system specific (non cross platform) API's would
mean a rewrite of the GUI code for each operating system that was to be
supported.

\subsection{Mobile solution for Android /
IOS}\label{mobile-solution-for-android-ios}

The production of a mobile application for either the Android or IOS
markets would be done with either two versions of the application as
discussed with the desktop versions above, or there are tools like
\href{https://www.xamarin.com/platform}{Xamarin}

\begin{verbatim}
"Deliver native Android, iOS, and Windows apps, using existing skills,
teams, and code."
*cite xamerin*
\end{verbatim}

This allows for applications to be built for a variety of mobile
platforms with a single code base in C\#. The ability to ``write once
run anywhere'' is a big selling point for Xamarin and others like it.
However it would still restrict users into using mobile only.

\subsection{Browser based solution}\label{browser-based-solution}

With modern advances in browser technology it has become more feasible
to create full desktop replacement applications as website-applications.
These websites are truly cross platform, they can be viewed on any
platform with access to a web browser. Desktop OS or mobile OS alike can
all use one of a multitude of recent browsers from many vendors. -
Chrome (Windows, Mac OSX, IOS, Android) - Safari (Mac OSX, IOS, Windows)
- Opera (Windows, Mac OSX, IOS \{opera-mini\}, Android \{opera-mini\}) -
Firefox (Windows, Mac OSX, IOS, Android)

There are many different libraries on offer to help develop applications
with javascript. However it is still possible to create more traditional
client-server applications and have just a simple lightweight javascript
free front end. These applications tend to be slower and more cumbersome
to use because of there need to communicate every action with a server
over http.

An alternative to the more traditional client-server architecture is the
now more popular single page javascript application. This merges the
lines between the user perceived differences of the desktop and browser
experience.

A single page javascript application is a single web page that functions
like a normal application. The application have all the features of a
desktop application, they can make use of windows, buttons, and
animations etc\ldots{}

External data is gathered in background http calls (called Ajax
requests) and does not necessitate a full page reload. The absence of
these full reloads and the flash of white as the page loads give the
user a much better experience.

\emph{cite}

Single Page Web Applications By Michael S. Mikowski and Josh C. Powell

\subsection{Reason for selecting
browser}\label{reason-for-selecting-browser}

For the application being built the choice of platform is between a
cross platform library for a native desktop application and a web based
solution. For reasons of experience with technology and programming
languages developing for the browser was selected as the platform.

The browser application will be accessible from any platform and device
and provides an easy way to update the application in the future.

There are many different libraries that can help when building a
javascript application in the browser. The libraries have some things in
common and other unique points that set them apart from each other. Most
of the libraries cover the standard principles of MVC ``Model View
Controller'' or some similar variant thereof. See
\href{https://en.wikipedia.org/wiki/List_of_JavaScript_libraries\#Web-application_related_.28MVC.2C_MVVM.29}{here}
for a community populated list on wikipedia.

\subsection{Javascript MVC library -
React.js}\label{javascript-mvc-library---react.js}

React.js is a fairly new open source library from facebook.

\begin{verbatim}
A JAVASCRIPT LIBRARY FOR BUILDING USER INTERFACES

*cite React.js website*
https://facebook.github.io/react/
\end{verbatim}

React.js brings the notions of pure functions to the problem of creating
and updating views. In React.js the headache of updating the view when
something happens in the application is handled automatically. This is
accomplished by having the view hierarchy be a result of applying a pure
function of the application state. This has the important consequence of
removing the need for the developer to tell the application how an
action should update the UI. The developer only needs to update the
state and React.js will work out what needs to change by calculating the
difference in the old and new view hierarchies.

\subsection{Future React Native}\label{future-react-native}

React.js has a new related project created by the same team called React
Native. React Native takes the principles of React.js and with slight
tweaks to the code the same code can run as a native application on both
IOS and Android.

This would provide a way to get the application onto mobile devices in a
native environment the users wouldn't even know that the application was
even running in the browser.

\section{Choice of interface - Notebook
metaphor}\label{choice-of-interface---notebook-metaphor}

The interface of any application need to be simple and easy to
understand. the more complicated the interface the more brainpower needs
to be dedicated to using it. The most successful user interfaces provide
simple intuitive ways for uses to do the actions they require.

\begin{verbatim}
Metaphors are the fundamental concepts, terms, and images
by which information is easily recognized, understood, and
remembered.
- Metaphor Design for User Interfaces, Marcus, Aaron
\end{verbatim}

\emph{cite} @inproceedings\{Marcus:1998:MDU:286498.286577, author =
\{Marcus, Aaron\}, title = \{Metaphor Design for User Interfaces\},
booktitle = \{CHI 98 Cconference Summary on Human Factors in Computing
Systems\}, series = \{CHI '98\}, year = \{1998\}, isbn =
\{1-58113-028-7\}, location = \{Los Angeles, California, USA\}, pages =
\{129--130\}, numpages = \{2\}, url =
\{http://doi.acm.org/10.1145/286498.286577\}, doi =
\{10.1145/286498.286577\}, acmid = \{286577\}, publisher = \{ACM\},
address = \{New York, NY, USA\}, keywords = \{Web, consumers, culture,
diversity, graphic design, icons, information design, metaphors,
multi-media, productivity tools, rhetoric, semantics, semi-otics,
symbols, user interfaces, visible language\}, \}

The metaphor chosen for the application is the notebook. A physical
notebook has pages, tabs etc\ldots{} These are easily understood by
anyone who knows what a notebook is. We can extend the notion of a
notebook with concepts from the wider world of books with indexes and
contents pages.

The application will take from this notion of the ``notebook metaphor''
and from there the application interface will be designed.

\emph{notebook image mapped to uml diagram of application}

\section{Detachment of the SQL interface through custom web
API}\label{detachment-of-the-sql-interface-through-custom-web-api}

Selecting the browser poses some specific problems for a SQL IDE. A SQL
IDE needs to have access to the SQL server in order to execute queries
and retrieve results. There is no support in any of the mainstream
browsers for direct integration with a SQL server database, although
they do have limited support \emph{cite web sql} for in browser
databases.

This therefore requires the production of some middleware to connect the
database to the application. The commonly used method for transferring
data in the browser is via a JSON web API. There are many methods for
connecting to a SQL server database however, the Microsoft documentation
\emph{cite} https://msdn.microsoft.com/en-us/library/ms162132.aspx
contains C\#, VB, C++ documentation and libraries for querying the
database. I have experience using the C\# conneciton methods before and
with this in mind the middleware was chosen to be a C\# website
providing a mapping from JSON to the database.

\emph{diagram of connection}
