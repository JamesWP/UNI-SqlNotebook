\chapter{Evaluation and Future
developments}\label{evaluation-and-future-developments}

\section{Testing the project}\label{testing-the-project}

Testing is an important part of developing an application. In order to achieve
its goals, the application must function correctly, with testing and
verification ensuring that each part of the system operates correctly.

The system was designed to be easily tested. The data store's interface
which provides the methods that the application uses was tested at the
beginning of each run of the application. The application would
initialise and then perform the testing procedure. The testing procedure
contained a simulation of a user performing some tasks, creating pages
and adding content etc..

After the system had finished running the user's commands the resulting
UI could be checked against the expected output and in this way
regression could be detected early. The JavaScript console also provides
an excellent way to monitor the application for errors at runtime and
debug to find their cause(s).

\section{Evaluation}\label{evaluation}

For the application to be successful the requirements and use cases need
to be evaluated against the application. The application will be
successful if each of the requirements are met and the use cases can be
shown to be completed by the users.

For my evaluation I presented the application to the users and asked them
to complete some simple tasks. The users had not seen the application
before and also had not seen any of the data stored before. This test
shows how easy the application is to learn and also provides
feedback on how real users interacted with the UI to find the
information stored in the system.

The testers were only told a bare minimum of what the application was
supposed to do. The application should allow for the organisation of
data and code for a SQL application.

Overall the testers were all able to find the various functions of
the application by navigating the familiar UI features like the burger
menu, ``X'' for closing windows. Finding the save options was
similarly found by them.

The tools for executing and saving the results from queries were also
quickly learned by the users. This feature requires the most ``clicks''
in the application and so could be viewed as the most complicated
feature however, the users rarely got lost and the actions flowed
naturally to the users.

The testers also made some comments on what features they would like to
see in the future for the application. These future features will be
discussed below.

\section{Future developments}\label{future-developments}

The application as it is can be used for daily work by SQL developers
and meets the requirements set out initially. However there are some
features that did not fit into the schedule of remaining time. These
features were not essential to completing a functional working
application however, they would make the application more useful.

\subsection{Multiple database
engines}\label{multiple-database-engines}

Currently the application only supports Microsoft SQL Server however
with the addition of more middle ware and a configurable settings window
the application could be made to talk to any database server.

This would enable the users to work on multiple systems across multiple
database providers. It would also open the application to the users that
solely use one of those databases.

\subsection{Query explanation}\label{query-explanation}

The middle ware supports a method to explain the objects referenced in a
query. This functionality could be used to provide code completions and
also more data when the objects are hovered over.

The meta-data could also be saved in the file when the page is saved.
This would then be available when the page was loaded. This could
provide extra context when looking into old queries.

\subsection{Different result formats}\label{different-result-formats}

The results from the queries are shown in a table. The results of
queries are just arrays of n-tuples but they could represent more
complex items like pi charts or folder hierarchies.

There could be some automatic classification of results to decide the
best way to display the results. Failing that, fall back onto manual
selection for the results presentation.

\subsection{Resizing windows}\label{resizable-windows}

The workspace shows the open pages and their controls. Currently the
application sizes these to have all the same width however, this might
not be optimal for some files. The developer has no control over how
they are presented. It would be a valuable feature of the system to be
able to reorder and resize the windows for each open page. This would
make the system more customizable for each user.
